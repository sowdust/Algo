%%%%%%%%%%%%%%%%%%%%%%%%%%%%%%%%%%%%%%%%%
% Structured General Purpose Assignment
% LaTeX Template
%
% This template has been downloaded from:
% http://www.latextemplates.com
%
% Original author:
% Ted Pavlic (http://www.tedpavlic.com)
%
% Note:
% The \lipsum[#] commands throughout this template generate dummy text
% to fill the template out. These commands should all be removed when 
% writing assignment content.
%
%%%%%%%%%%%%%%%%%%%%%%%%%%%%%%%%%%%%%%%%%

%----------------------------------------------------------------------------------------
%	PACKAGES AND OTHER DOCUMENT CONFIGURATIONS
%----------------------------------------------------------------------------------------

\documentclass{article}

\usepackage{fancyhdr} % Required for custom headers
\usepackage{lastpage} % Required to determine the last page for the footer
\usepackage{extramarks} % Required for headers and footers
\usepackage{graphicx} % Required to insert images
\usepackage{lipsum} % Used for inserting dummy 'Lorem ipsum' text into the template

% Margins
\topmargin=-0.45in
\evensidemargin=0in
\oddsidemargin=0in
\textwidth=6.5in
\textheight=9.0in
\headsep=0.25in 

\linespread{1.1} % Line spacing

% Set up the header and footer
\pagestyle{fancy}
\lhead{\hmwkAuthorName} % Top left header
\chead{\hmwkClass\ : \hmwkTitle} % Top center header
\rhead{\firstxmark} % Top right header
\lfoot{\lastxmark} % Bottom left footer
\cfoot{} % Bottom center footer
\rfoot{Pagina\ \thepage\ di\ \pageref{LastPage}} % Bottom right footer
\renewcommand\headrulewidth{0.4pt} % Size of the header rule
\renewcommand\footrulewidth{0.4pt} % Size of the footer rule

\setlength\parindent{0pt} % Removes all indentation from paragraphs

%----------------------------------------------------------------------------------------
%	DOCUMENT STRUCTURE COMMANDS
%	Skip this unless you know what you're doing
%----------------------------------------------------------------------------------------

% Header and footer for when a page split occurs within a problem environment
\newcommand{\enterProblemHeader}[1]{
%\nobreak\extramarks{#1}{#1 continued on next page\ldots}\nobreak
%\nobreak\extramarks{#1 (continued)}{#1 continued on next page\ldots}\nobreak
}

% Header and footer for when a page split occurs between problem environments
\newcommand{\exitProblemHeader}[1]{
%\nobreak\extramarks{#1 (continued)}{#1 continued on next page\ldots}\nobreak
%\nobreak\extramarks{#1}{}\nobreak
}

\setcounter{secnumdepth}{0} % Removes default section numbers
\newcounter{homeworkProblemCounter} % Creates a counter to keep track of the number of problems

\newcommand{\homeworkProblemName}{}
\newenvironment{homeworkProblem}[1][Problem \arabic{homeworkProblemCounter}]{ % Makes a new environment called homeworkProblem which takes 1 argument (custom name) but the default is "Problem #"
\stepcounter{homeworkProblemCounter} % Increase counter for number of problems
\renewcommand{\homeworkProblemName}{#1} % Assign \homeworkProblemName the name of the problem
\section{\homeworkProblemName} % Make a section in the document with the custom problem count
\enterProblemHeader{\homeworkProblemName} % Header and footer within the environment
}{
\exitProblemHeader{\homeworkProblemName} % Header and footer after the environment
}

\newcommand{\problemAnswer}[1]{ % Defines the problem answer command with the content as the only argument
\noindent{\begin{minipage}{0.98\columnwidth}#1\end{minipage}} % Makes the box around the problem answer and puts the content inside
}

\newcommand{\homeworkSectionName}{}
\newenvironment{homeworkSection}[1]{ % New environment for sections within homework problems, takes 1 argument - the name of the section
\renewcommand{\homeworkSectionName}{#1} % Assign \homeworkSectionName to the name of the section from the environment argument
\subsection{\homeworkSectionName} % Make a subsection with the custom name of the subsection
\enterProblemHeader{\homeworkProblemName\ [\homeworkSectionName]} % Header and footer within the environment
}{
\enterProblemHeader{\homeworkProblemName} % Header and footer after the environment
}
   
%----------------------------------------------------------------------------------------
%	NAME AND CLASS SECTION
%----------------------------------------------------------------------------------------

\newcommand{\hmwkTitle}{Compito\ \#10} % Assignment title
\newcommand{\hmwkDueDate}{ } % Due date
\newcommand{\hmwkClass}{Algoritmi} % Course/class
\newcommand{\hmwkClassTime}{ } % Class/lecture time
\newcommand{\hmwkClassInstructor}{ } % Teacher/lecturer
\newcommand{\hmwkAuthorName}{Mattia Vinci} % Your name

%----------------------------------------------------------------------------------------
%	TITLE PAGE
%----------------------------------------------------------------------------------------

\title{
\vspace{2in}
\textmd{\textbf{\hmwkClass:\ \hmwkTitle}}\\
\normalsize\vspace{0.1in}\small{Due\ on\ \hmwkDueDate}\\
\vspace{0.1in}\large{\textit{\hmwkClassInstructor\ \hmwkClassTime}}
\vspace{3in}
}

\author{\textbf{\hmwkAuthorName}}
\date{} % Insert date here if you want it to appear below your name

%----------------------------------------------------------------------------------------

\begin{document}

%\maketitle

%----------------------------------------------------------------------------------------
%	TABLE OF CONTENTS
%----------------------------------------------------------------------------------------

%\setcounter{tocdepth}{1} % Uncomment this line if you don't want subsections listed in the ToC

%\newpage
%\tableofcontents
%\newpage

%----------------------------------------------------------------------------------------
%	PROBLEM 1
%----------------------------------------------------------------------------------------

% To have just one problem per page, simply put a \clearpage after each problem

\begin{homeworkProblem}[Confronto algoritmi di ordinamento]

Usando il linguaggio C, sono stati implementati i seguenti algoritmi di ordinamento:

\begin{itemize}
  \item Selection Sort
  \item Insertion Sort
  \item Merge Sort
  \item Quick Sort (versione Hoare)
  \item Heap Sort
\end{itemize}

Tramite un altro programma C, si sono confrontate le performance (dal punto di vista temporale) dei suddetti algoritmi su input di diverse dimensioni; i risultati (calcolati in millisecondi) sono stati salvati in un file \emph{cvs} dal quale si sono poi tratti dei grafici esplicativi.

I vettori da ordinare forniti in input sono stati riempiti in maniera sia ordinata sia pseudo-casuale (in quest'utlimo caso utilizzando la libreria standard C).


\vspace{10pt}


\begin{homeworkSection}{}
\problemAnswer{ % Answer



\begin{center}
\includegraphics[width=0.75\columnwidth]{ordinati} % Example image
\end{center}

Per prima cosa si sono confrontati tutti gli algoritmi dando loro in pasto degli array di crescente lunghezza gia' ordinati. Come ci si aspettava l'\emph{heapsort} non ha un caso molto peggiore rispetto al medio in tali condizioni, come invece accade al \emph{quicksort}.

}

\end{homeworkSection}





\begin{homeworkSection}{}
\problemAnswer{ % Answer



\begin{center}
\includegraphics[width=0.75\columnwidth]{ordinati2} % Example image
\end{center}

Confrontando pero' solo i tre algoritmi dalle migliori prestazioni nel caso di array ordinati, ovvero l' \emph{InsertionSort}, il \emph{MergeSort ottimizzato} (cioe' che non compie l'operazione di merge ove non necessario) e l'\emph{HeapSort} e' immediato notare come i primi due si comportino "molto meglio" del secondo: essi sono infatti lineari in questo miglior caso, mentre l'\emph{HeapSort} rimane pseudo-lineare.

}

\end{homeworkSection}




\begin{homeworkSection}{}
\problemAnswer{ % Answer



\begin{center}
\includegraphics[width=0.75\columnwidth]{tuttiInsieme} % Example image
\end{center}

Usando degli array non ordinati si e' confermata l'appartenenza della complessita' temporale dell'\emph{HeapSort} alla famiglia di quelle di \emph{Mergesort} e \emph{Quicksort}



}

\end{homeworkSection}





\begin{homeworkSection}{}
\problemAnswer{ % Answer



\begin{center}
\includegraphics[width=0.75\columnwidth]{QvsHvsM} % Example image
\end{center}

Confrontando infine questi tre algoritmi piu' performanti l'\emph{HeapSort} risulta distaccarsi notevolmente (in peggio) sia dal \emph{QuickSort} che dal \emph{MergeSort}, che si rivelano invece tra loro molto simili, con un leggero vantaggio del primo rispetto al secondo.

Mentre e' nota la migliore prestazione empirica del \emph{QuickSort} rispetto agli altri algoritmi di ordinamento, il vantaggio del \emph{MergeSort} rispetto all'\emph{HeapSort} era meno previsto.

Tentando di dare una spiegazione, si potrebbe ipotizzare che il \emph{MergeSort} abbia una maggiore frequenza di hit nella cache rispetto a quella dell'\emph{HeapSort}.



}

\end{homeworkSection}







\end{homeworkProblem}

%----------------------------------------------------------------------------------------

\end{document}
